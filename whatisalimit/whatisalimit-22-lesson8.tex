\documentclass{ximera}

\title{What is a limit?}

\newenvironment{objectives}{\begin{remark}\textbf{Objectives}\\}{\end{remark}}

\begin{document}
\begin{abstract}
\end{abstract}

\subsection{Famous functions}

\begin{theorem}
    The following functions are continuous on their natural domains, for $k$ a real number and $b$ a positive real number.
    \begin{itemize}
        \item Constant function \[f(x)=k\]
        \item Identity function \[f(x)=x\]
        \item Power function \[f(x)=x^b\]
        \item Exponential function \[f(x)=b^x\]
        \item Logarithmic function \[f(x)=\log_b(x)\]
        \item Sine and cosine \[f(x)=\sin(x), \qquad f(x) = \cos(x)\]
    \end{itemize}
\end{theorem}

We can now take limits of these functions without having to guess by evaluating them at nearby numbers. We know that for all values of $a$ in the domain of these functions, we can find the limit as $x$ approaches $a$ just by plugging $a$ into the function!
\[
    \lim_{x \to a} f(x) = f(a)
\]

\end{document}