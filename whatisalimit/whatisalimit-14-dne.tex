\documentclass{ximera}

\title{DNE}

\newenvironment{objectives}{\begin{remark}\textbf{Objectives}\\}{\end{remark}}

\begin{document}
\begin{abstract}
\end{abstract}

\maketitle

\section{Limits which ``Do Not Exist''}

Sometimes, a limit may not exist.

\begin{explanation}
    A limit does not exist if $f(x)$ does not approach a single value as $x$ approaches $a$.
\end{explanation}

We use the notation DNE (which stands for ``Does Not Exist'') when a limit does not exist.

Use the links below to explore three scenarios in which the limit does not exist (all three should be required; not sure how to do this in Ximera). After you have explored all three, continue to the next page to complete the assessment.

\subsection{Oscillations}
\begin{center}
    \youtube{EoT5FZdt6qs}
\end{center}
\subsection{One-sided limits do not match}
\begin{center}
    \youtube{KI5tjq2yrcI&t=131s}
\end{center}
\subsection{Infinite limits}
\begin{center}
    \youtube{gCuh1DmvvQE}
\end{center}


\end{document}