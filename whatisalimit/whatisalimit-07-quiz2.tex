\documentclass{ximera}

\title{What is a limit?}

\newenvironment{objectives}{\begin{remark}\textbf{Objectives}\\}{\end{remark}}

\begin{document}
\begin{abstract}
\end{abstract}

\subsection{Quiz}

\begin{question}  
Which of the following statements is \underline{true} regarding the relationship between the limit as $x$ goes to $a$ of $f(x)$ and $f(a)$?  That is, the relationship between the limit and the function value at the point $x=a$. (You should have 2 attempts; not sure if there's a way to do this in Ximera.)
\begin{multipleChoice}  
\choice{The limit of $f(x)$ as $x$ goes to $a$ and the function value $f(a)$ always give the same value because the function must be approaching its function value.}
\choice{The limit of $f(x)$ as $x$ goes to $a$ and the function value $f(a)$ always give different values since they are by definition different quantities.}
\choice[correct]{The limit of $f(x)$ as $x$ goes to $a$ and the function value $f(a)$ are by definition different quantities. They may or may not have the same numerical value, depending on the behavior of $f(x)$ near $x=a$.}  
\end{multipleChoice}  

% \begin{exploration}
\begin{explanation}
    That's right! The limit of $f(x)$ as $x \to a$ and the function value $f(a)$ are by definition different quantities. They may or may not have the same numerical value, depending on the behavior of $f(x)$ near $x=a$.
% \end{exploration}
\end{explanation}
\end{question}

\end{document}