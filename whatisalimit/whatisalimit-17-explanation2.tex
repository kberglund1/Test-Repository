\documentclass{ximera}

\title{What is a limit?}

\newenvironment{objectives}{\begin{remark}\textbf{Objectives}\\}{\end{remark}}

\begin{document}
\begin{abstract}
\end{abstract}

\subsection{Requirements for continuity}

In order for the function $f(x)$ to be continuous at a point $x=a$, the following conditions must be met:
\begin{enumerate}
    \item $f(a)$ is defined
    \item $\lim_{x \to a} f(x)$ exists
    \item $\lim_{x \to a} f(x) = f(a)$ (the function value and limit are equal at $a$).
\end{enumerate}

\begin{explanation}
    \textbf{Intuitive vs. Formal Definitions.} Intuitively, the graph of $f$ is continuous at the point $a$ if the graph near $a$ can be traced ``without lifting your pencil''. While this way of thinking about continuity can be helpful to us in understanding the concept, it is not the mathematical definition. If you are asked to show a function is continuous at a point, you must use the mathematical definition given above!
\end{explanation}

\end{document}