\documentclass{ximera}

\title{Finding the Domain of Functions}
\author{Kenneth Berglund}
\begin{document}
\begin{abstract}
\end{abstract}
\maketitle

\section{Definition}

The domain of a function $f$ is the set of all possible inputs to the function. It excludes the points at which a function is undefined. 

\begin{example}
If $f$ is defined by 

$$f(x) = \frac{1}{x - 2}$$

find the domain of $f$. 

\begin{explanation}
Since plugging in $x = 2$ makes the denominator of $f(x)$ equal to zero, 2 is not in the domain of $f$. All other real numbers are okay to plug in, so the domain of $f$ is all $x \ne 2$. In interval notation, this is $(-\infty, 2) \cup (2, \infty)$.

Note from the graph below that the function is not defined for $x = 2$, which matches what we found algebraically. A link to the Desmos graph can be found \link[here]{https://www.desmos.com/calculator/iljw8bk2ne}. 

\begin{center}
\desmos{iljw8bk2ne}{800}{600}
\end{center}
\end{explanation}
\end{example}

\section{What causes functions to be undefined?}
There are three main ways functions can be undefined. Follow the links below for explanations and examples of each type.

\link[Denominators]{domaindenominators}

\link[Even Roots]{domainevenroots}

\link[Logarithms]{domainlogarithms}

When you're done exploring these, try out the \link[mixed exercises]{domainmixedexamples}.
\end{document}