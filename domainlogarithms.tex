\documentclass{ximera}

\title{ Logarithms}
\author{Kenneth Berglund}
\begin{document}
\begin{abstract}
Discusses how when there is a negative or zero inside a logarithm, the function is undefined. 
\end{abstract}
\maketitle

\section{Logarithms}
You might not have encountered logarithms yet in your course, so if you haven’t, feel free to skip this section until you get to them.

The domain of $\log_b(x)$ (for any base $b$) is all positive real numbers. If your function has a variable inside a logarithm function, \textbf{you risk having a non-positive number inside the logarithm}, which would make your function undefined. Your task is to find out which values of $x$ make the input to the logarithm positive. The domain will be those values.

\textbf{Be careful! The base of the logarithm does not matter when calculating the domain.}

\subsection{Why?}
The function $f(x) = \log_b(x)$ has domain $x > 0$, so any expression with a non-positive number inside the logarithm is undefined. 

\subsection{Example}
The function $f(x) = \ln(x - 1)$ has domain $(1, \infty)$, since those values of $x$ make the input to the log positive. To find this, we set $x - 1 > 0$ and solved for $x$.

In the below graph, we can see that the function is not defined for $x \le 1$. \link[Here]{https://www.desmos.com/calculator/ym6ox91gda} is a link to the Desmos graph. 
\begin{center}
\desmos{ym6ox91gda}{800}{600}
\end{center}

Feel free to check your solutions using Desmos to see where the function is defined. We also provide solutions. \textbf{Please try each problem before looking at the solutions, and only resort to the solution when you get stuck.} 

\subsection{More Examples}
Find the domain of $f$ for:

\begin{enumerate}
	\item $f(x) = \log(x - 3)$
	\item $f(x) = \ln(2|x| + 1)$
	\item $f(x) = \ln(77 - x)$
	\item $f(x) = \log(4x + 6)$
\end{enumerate}

\begin{explanation}
\begin{enumerate}
	\item We set the input to the logarithm greater than zero: $x - 3 > 0$. Solving this, we get $x > 3$, which is our domain. In interval notation, this is $(3, \infty)$. 

	\item We set the input to the logarithm greater than zero: $2|x| + 1 \ge 0$.

To solve this, we first isolate the absolute value: $|x| \ge -\frac{1}{2}$. This is always true, since an absolute value measures distance, which is never negative. Therefore, our domain is all real numbers, or in interval notation $(-\infty, \infty)$.
	\item We set the input to the logarithm greater than zero: $77-x> 0$. Solving this, we get $x < 77$, which is our domain. In interval notation, this is $(-\infty, 77)$.   

	\item We set the input to the logarithm greater than zero: $4x + 6> 0$. Solving this, we get $x > -\frac{3}{2}$, which is our domain. In interval notation, this is $\left(-\frac{3}{2}, \infty\right)$. 
\end{enumerate}
\end{explanation}

\link[Back to the main page]{domainmain}

\end{document}