\documentclass{ximera}

\title{Mixed Examples}
\author{Kenneth Berglund}
\begin{document}
\begin{abstract}
Discusses how to find the domain of  a function when the function contains multiple places it could be undefined. 
\end{abstract}
\maketitle

\section{Mixed Examples}
For these problems, go through all possible ways the function could be undefined, then combine the restrictions that you get.

\subsection{Worked Example}
Find the domain of $f$ for $$f(x) = \frac{\sqrt{2x - 3}}{x - 7}.$$

\begin{explanation}
First, note that $x \ne 7$, since that makes the denominator zero.

Then, we set the radicand greater than or equal to zero:

\begin{align*}2x-3&\ge 0 \\ 2x & \ge 3 \\ x & \ge \frac{3}{2}. \end{align*}

Combining these restrictions gives us the domain $x \ge \frac{3}{2}$ with $x \ne 7$. In interval notation, this is

$$\left[\frac{3}{2}, 7\right) \cup (7, \infty).$$
\end{explanation}
\subsection{More Examples}
Feel free to check your solutions using Desmos to see where the function is defined. We also provide solutions, which can be viewed by clicking on the arrow to the right of the problem. \textbf{Please try each problem before looking at the solutions, and only resort to the solution when you get stuck.} 

Find the domain of $f$ for:

\begin{enumerate}
	\item $f(x) = \frac{\sqrt{3x - 4}}{x - 2}$
		\begin{explanation}
			\begin{expandable}
First, note that $x \ne 2$, since that makes the denominator zero.

Then, we set the radicand greater than or equal to zero: \begin{align*}3x-4&\ge 0 \\ 3x & \ge 4 \\ x & \ge \frac{4}{3}. \end{align*}

Combining these restrictions gives us the domain $x \ge \frac{4}{3}$ with $x \ne 2$. In interval notation, this is

$$\left[\frac{4}{3}, 2\right) \cup (2, \infty).$$
			\end{expandable}
		\end{explanation}
	\item $f(x) = \frac{1}{\sqrt{x - 6}}$
		\begin{explanation}
			\begin{expandable}

First, note that $x \ne 6$, since that makes the denominator zero.

Then, we set the radicand greater than or equal to zero: \begin{align*}x-6&\ge 0 \\ x & \ge 6. \end{align*}

Combining these restrictions gives us the domain $x \ge 6$ with $x \ne 6$, or $x > 6$. In interval notation, this is $(6, \infty)$.
			\end{expandable}
		\end{explanation}
	\item $f(x) = \frac{x - 2}{\sqrt{x + 4}}$
		\begin{explanation}
			\begin{expandable}
First, note that $x \ne -4$, since that makes the denominator zero.

Then, we set the radicand greater than or equal to zero: \begin{align*}x + 4&\ge 0 \\ x & \ge -4. \end{align*}

Combining these restrictions gives us the domain $x \ge -4$ with $x \ne -4$, or $x > -4$. In interval notation, this is $(-4, \infty)$.

Notice that the numerator does not introduce any restrictions on the domain. 
			\end{expandable}
		\end{explanation}
	\item $f(x) = \frac{\sqrt{1-x}}{x^2 + 3x - 40}$
		\begin{explanation}
			\begin{expandable}
First let's find what makes the denominator zero, i.e. solve $x^2 + 3x - 40 = 0$. We can factor to obtain $(x - 5)(x + 8) = 0$, so $x = 5$ and $x = -8$. This means that the domain has restrictions $x \ne 5$ and $x \ne -8$. 

Then, we set the radicand greater than or equal to zero: \begin{align*}1 - x&\ge 0 \\ 1 & \ge x. \end{align*}

Combining all these restrictions gives us $x \le 1$, $x \ne 5$ and $x \ne -8$. Note that the restriction $x \le 1$ automatically takes care of the restriction $x \ne 5$. 

In interval notation, our domain is $(- \infty, -8) \cup (-8, 1]$. 
			\end{expandable}
		\end{explanation}
	\item $f(x) = \frac{x^2 - 4}{5 - x}$
		\begin{explanation}
			\begin{expandable}
First, note that $x \ne 5$, since that makes the denominator zero.

Then, we set the radicand greater than or equal to zero: \begin{align*}5 - x&\ge 0 \\ 5 & \ge x. \end{align*}

Combining these restrictions gives us the domain $x \le 5$ with $x \ne 5$, or $x < 5$. In interval notation, this is $(-\infty, 5)$.

Note that the numerator does not introduce any restrictions on the domain. 
			\end{expandable}
		\end{explanation}
	\item $f(x) = \frac{1}{\log(x + 2)}$
		\begin{explanation}
			\begin{expandable}
First, we find where the denominator equals zero, i.e. solve $\log(x + 2) = 0$. Exponentiating both sides with a base of 10 yields \begin{align*}10^{\log(x + 2)} & = 10^0 \\ x + 2 & = 1 \\ x & = -1. \end{align*}

This means that the domain has the restriction $x \ne -1$. 

Next, we set the input of the logarithm greater than zero: $x + 2 > 0$. This yields $x > -2$.

Therefore, our domain is $x > -2$ with $x \ne -1$, or in interval notation, $(-2, -1) \cup (-1, \infty)$. 
			\end{expandable}
		\end{explanation}
	\item $f(x) = \frac{\log(4 - x)}{\sqrt{x - 2}}$
		\begin{explanation}
			\begin{expandable}
First, note that $x \ne 2$, since that makes the denominator zero.

Next, we set the input of the logarithm greater than zero: $4 - x > 0$. This yields $4 > x$, or $x < 4$.

Then, we set the radicand greater than or equal to zero: \begin{align*}x - 2&\ge 0 \\ x & \ge 2. \end{align*}

This means our domain has restrictions $x \ne 2$, $x < 4$, and $x \ge 2$. Putting these together, we see that our domain is $(2, 4)$. 
			\end{expandable}
		\end{explanation}
\end{enumerate}



\link[Back to the main page]{domainmain}

\end{document}