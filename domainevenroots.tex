\documentclass{ximera}

\title{Even Roots}
\author{Kenneth Berglund}
\begin{document}
\begin{abstract}
Discusses how when there is a negative underneath an even root, the function is undefined. 
\end{abstract}
\maketitle

\section{Even Roots}
If your function has a variable underneath an even root (a square root, fourth root, sixth root, etc.) \textbf{you risk having a negative number underneath the square root}, which would make your function undefined. Your task is to find out which values of $x$ make the radicand (the expression underneath the square root) non-negative. The domain will be those values.

\subsection{Why?}
The function $f(x) = \sqrt{x}$ has domain $x \ge 0$, so any expression with a negative underneath the square root is undefined. 

\subsection{Example}
The function $f(x) = \sqrt{1 - x}$ has domain $(-\infty, 1]$, since those values of $x$ make the radicand non-negative. To find this, we set $1 - x \ge 0$ and solved for $x$. 

Note that in the graph below, we can visually see how the function only takes values when $x \le 1$. \link[Here]{https://www.desmos.com/calculator/m7k2scsnpa} is a link to the Desmos graph. 
\begin{center}
\desmos{m7k2scsnpa}{800}{600}
\end{center}

Feel free to check your solutions using Desmos to see where the function is defined. We also provide solutions. \textbf{Please try each problem before looking at the solutions, and only resort to the solution when you get stuck.} 

\subsection{More Examples}
Find the domain of $f$ for:

\begin{enumerate}
	\item$f(x) = \sqrt{x - 2}$
		\begin{expandable}
			Solution:

We set the radicand greater than or equal to zero: $x - 2 \ge 0$. Solving this, we get $x \ge 2$, which is our domain. In interval notation, this is $[2, \infty)$. 
		\end{expandable}
	\item $f(x) = \frac{\sqrt{x - 3}}{5}$
		\begin{expandable}
			Solution:

We can ignore the denominator here, since there is no variable in the denominator. We set the radicand greater than or equal to zero: $x - 3 \ge 0$. Solving this, we get $x \ge 3$, which is our domain. In interval notation, this is $[3, \infty)$. 
		\end{expandable}
	\item $f(x) = \sqrt[4]{x + 3}$
		\begin{expandable}
			Solution:

Note that \textbf{any} even root can cause a function to be undefined at points. We set the radicand greater than or equal to zero: $x + 3 \ge 0$. Solving this, we get $x \ge -3$, which is our domain. In interval notation, this is $[-3, \infty)$.  
		\end{expandable}
	\item $f(x) = 3\sqrt{x - 7}$
		\begin{expandable}
			Solution:

The coefficient in front of the radical does not affect the domain. We set the radicand greater than or equal to zero: $x -7 \ge 0$. Solving this, we get $x \ge 7$, which is our domain. In interval notation, this is $[7, \infty)$. 
		\end{expandable}
	\item $f(x) = \sqrt{2|x| + 3}$
		\begin{expandable}
			Solution:

We set the radicand greater than or equal to zero: $2|x| - 3 \ge 0$.

To solve this, we first isolate the absolute value: $|x| \ge \frac{3}{2}$. This tells us $x \ge \frac{3}{2}$ or $x \le -\frac{3}{2}$ (review how to solve absolute value inequalities if you need to). In interval notation, our domain is $\left(-\infty, -\frac{3}{2}\right) \cup \left(\frac{3}{2}, \infty\right)$.
		\end{expandable}
\end{enumerate}


\link[Back to the main page]{domainmain}

\end{document}